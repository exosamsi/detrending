\documentclass[12pt,letterpaper]{article}

\newcommand{\warning}[1]{\texttt{#1}}
\newcommand{\project}[1]{\textsl{#1}}
\newcommand{\Kepler}{\project{Kepler}}
\newcommand{\given}{\,|\,}
\newcommand{\setof}[1]{\left\{{#1}\right\}}
\newcommand{\datum}{Y}
\newcommand{\data}{\setof{\datum_n}_{n=1}^N}
\renewcommand{\time}{t}
\newcommand{\exofn}{q}
\newcommand{\exopars}{\omega}
\newcommand{\flux}{F}
\newcommand{\starpars}{\alpha}
\newcommand{\noise}{e}
\newcommand{\variance}{\sigma^2}
\newcommand{\hyperpars}{\varphi}
\newcommand{\normal}{N}
\newcommand{\mean}{\mu}

\begin{document}

\section*{A partially marginalized likelihood for exoplanet inference}

\noindent
David W. Hogg (NYU) \\
\textit{with help from Paul Baines (Davis) and everyone else at SAMSI \Kepler}

\paragraph{disclaimer:}
\warning{This document is a draft, and not yet ready for public consumption.
In particular, any use of the content in this document would represent a serious lapse in judgement.}

\paragraph{abstract:}
We present a likelihood function for the time-series photometry
that makes up a lightcurve from the \Kepler\ mission (or similar experiment).
The function includes parameters that describe stochastic stellar variability
and parameters that describe periodic exoplanet (or other companion) transits.
The stellar variability model is a Gaussian Process in a wavelet basis
that is capable of modeling non-trivial time correlations with a diagonal covariance matrix.
These choices make it possible to \emph{marginalize out} all stellar-variability parameters,
leaving the user with a flexible likelihood function parameterized only by exoplanet parameters.
We show \warning{something non-trivial}.

\section{generalities}

There are $N$ observations $\datum_n$ of a single star taken at times $\time_n$.
The model is
\begin{eqnarray}
\datum_n &=& [1 - \exofn(t_n\given\exopars)]\,\flux(t_n\given\starpars) + \noise_n
\quad ,
\end{eqnarray}
where
$\exofn(\cdot\given\exopars)$ is a function
that describes the attenuation of the starlight caused by the transiting exoplanet,
$\exopars$ is a blob of parameters describing an exoplanet's size and orbit,
$\flux(\cdot\given\starpars)$ is a function
that describes the apparent brightness of the star as a function of time,
$\starpars$ is a blob of parameters describing the stellar mean flux and variability,
and $\noise_n$ is the noise contribution to the $n$th datum
(coming from photon and read noise, among other things).
This description---plus a model for the noise---will lead to a justifiable likelihood function.
If we model the noise contributions as being Gaussian and independent
with known variances $\variance_n$, the likelihood function becomes
\begin{eqnarray}
p(\data\given\exopars,\starpars,\hyperpars)
  &=& \prod_{n=1}^N p(\datum_n\given\exopars,\starpars,\hyperpars)
\\
p(\datum_n\given\exopars,\starpars,\hyperpars)
  &=& \normal(\datum_n\given\mean_n,\variance_n)
\\
\mean_n
  &\equiv& [1 - \exofn(t_n\given\exopars)]\,\flux(t_n\given\starpars)
\quad ,
\end{eqnarray}
where
$\data$ is the set of all data,
$\hyperpars$ is an enormous blob of hyperparameters
that includes all our decision-making and assumptions (and more, soon),
the product in the likelihood encodes our ``independent noise'' assumption,
and $\normal(x\given m,V)$ is the Gaussian for $x$ with mean $m$ and variance $V$,

By assumption, we care only about the exoplanet parameters $\exopars$
and not in the least about the star parameters $\starpars$.


\end{document}
